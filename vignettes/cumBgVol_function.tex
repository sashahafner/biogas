%% Next 2 lines needed for non-Sweave vignettes
%\VignetteEngine{knitr::knitr} 
%\VignetteIndexEntry{Calculating methane and biogas production and production rates using volumetric methods}

\documentclass{article}

%%\usepackage[version=3]{mhchem} %chemical formulas
\usepackage[colorlinks = true, urlcolor = blue]{hyperref} % Must be loaded as the last package

\begin{Schunk}
\begin{Sinput}
> library(knitr)
> #opts_chunk$set(cache=FALSE,tidy=FALSE,highlight=FALSE)
> opts_chunk$set(cache = FALSE, tidy = FALSE, fig.align = "center")
> library(biogas)
>   options(width=75)
\end{Sinput}
\end{Schunk}

\title{Calculating methane and biogas production and production rates using volumetric  methods}
\author{Nanna Løjborg and Sasha D. Hafner (\texttt{sasha.hafner@eng.au.dk})}

\usepackage{Sweave}
\begin{document}
\input{cumBgVol_function-concordance}

\maketitle

\section{Introduction}
Biochemical methane potential (BMP) has become an important number in the biogas industri, as it can reveal essential knowledge of several factors of concern when producing biogas, such as substrate and inoculum behavior, or more process related variables (temp., pres., stirring, etc.). It is commonly used to determine the methane potential and biodegradability of a given substrate. A newly developed biogas package address issues with time-consuming calculations and lack of reproducible among laboratories for obtaining BMP (ref. biogas package or help file). The biogas package consists of ten function including cumBg(), which is used to calculate cumulative production of biogas and methane (CH$_4$) and production rates with either volumetric, manometric, gravimetric or gas density methods. These production values and rates can be furhter used to calculate BMP. cumBg() is a large and rather complicated function, which requires some proficiency in R for use. ##NTS: might be preferable to avoid including cumBg(), as these new functions should replace cumBg() at some point. 
We developed a function that only evaluate volumetric measurements to simplify functions within the biogas package, making it serviceable for public less experinced users. 
This document provides a brief description of the volumetric biogas calculation function (cumBgVol) for new users.
We have assumed that readers are familiar with biogas data collection, the biogas package and R.

\section{Overview of the function}
cumBgVol() is a ''high-level'' function within the biogas package. The purpose of cumBgVol() is to convert volume data collected in the laboratory to cumulative biogas and CH$_4$ production and to calculate production rates. The function can handle data from any number of bottles. For simple operations (e.g. interpolation and standardization of biogas volume) cumBgVol() is supported by calls to external low-level functions (Table \ref{tab:externalfunctionsummary}). The low-level functions are straight-forward to use, and details can be found in their individual help files.
This document describes the use of cumBgVol(). 

\begin{table}[h!]
  \begin{center}
  \caption{Operations done with the low-level functions in cumBgVol(). All functions are vectorized. See help files for more details.}
  \label{tab:lowfunctionsummary}
  \vspace{3pt}
  
  \begin{tabular}{ll}
    \hline
    Operation                                    &   Function \\
    \hline
    Standardize gas volume                       &   \texttt{stdVol()} \\
    Interpolate composition etc.                 &   \texttt{interp()} \\
    Structurize and sort data                    &   \texttt{cumBgDataPrep()} \\   
		\hline
  \end{tabular}
  \end{center}
\end{table}

In general, cumBg* functions are compiled of four sections: check arguments, restructuring and sorting data, interpolation if needed, and biogas standardization and calculations. Restructuring and sorting data and interpolation are handled by the external functions interp() and cumBgDataPrep(), respectively. From interp() gas composition, cumulative biogas production, and other variables can be interpolated to a specified time if required. From cumBgDataPrep() 'wide' and 'long' data structure are restructured to 'longcombo' data, which is required for cumBgVol() to further calculate cumulative biogas and CH$_4$ production and production rates. Additionally, data is sorted, headspace is added if provided, and composition data is corrected if it seems to be a percentage. If data of concern are mixed (interval and cumulative, \texttt{empty.name = TRUE}), these will be sorted and biogas volume standardized within cumBgDataPrep() to obtain interval data only. Subsequently, the now restructured and sorted data is standardized in cumBgVol() by an external function called stdVol(), if not already standardized. 
Two methods are commonly used to evaluate volumetric biogas measurements. Method 1 is based on normalized CH4 concentrations, whereas method 2 accounts for the actual CH4 in the bottle headspace. Both methods are available through cumBgVol() and results is expected to be independent of method. The examples below describe cumulative biogas calculation by volumetric method 1 on three datasets of different structures (wide, long, and longcombo). 
All external functions are within the biogas package
##NTS: Might be beneficial to describe both methods. Ensure it is method 1 in all examples. Also, might make more sense to have longcombo as the first example, as this i default.

\section{Examples:calculation of cumulative production of biogas and CH$_4$ and production rates using a volumetric calculation method}
Calculation of cumulative biogas and CH$_4$ production and production rates, typically requires two data frames: Biogas quantity (volume measurements) and biogas composition (CH$_4$ fraction).
Input data may be structured in one of three ways: ``long'', ``wide'', or ``longcombo''. Default is "longcombo". All inputs are accepted, but the volumetric calculation methods within cumBgVol() only process ''longcombo'' data structure. ''wide'' and ''long'' data are restructured internally by the low-level function cumBgDataPrep(). In the following examples all three data structures will be addressed.  

\newpage
\subsection{'wide' data structure}
In this example, we will use a wide structured example dataset included in the biogas package, having the data frame \texttt{feedVol} for biogas volumes. BMP measurement data are from a batch test carried out on animal feed ingredients along with cellulose as a control. 
The experiment included 12 batch bottles:
\begin{itemize}
  \item Three bottles with inoculum only (BK)
  \item Three bottles with cellulose and inoculum (CEL)
  \item Three bottles with animal feed ingredient 1 and inoculum (SC)
  \item Three bottles with animal feed ingredient 2 and inoculum (SD)
\end{itemize}

A typical automated volumetric method called AMPTS II was used to measure biogas production: an online, standardized lab-measurement platform for BMP tests. Applying AMPTS II, the measured volumes are standardized and the composition is 100\% methane. Therefore, the \texttt{comp}) argument is set to 1 when calling the cumBgVol function. Furthermore, pressure is set to a fixed value (atmospheric) and temperature to 0°C. 
\begin{Schunk}
\begin{Sinput}
> #library(biogas)
>   
> #data("feedVol")
> 
> dim(feedVol)